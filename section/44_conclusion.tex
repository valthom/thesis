 \chapter*{Conclusion}

In this thesis, we have explored the roles and challenges presented by noise in reinforcement learning and optimization algorithms. We have investigated the impact of stochasticity in the learning process and presented four significant contributions that address various aspects of this issue.

Our first two contributions focused on reinforcement learning in unknown environments, examining how we can design algorithms that leverage the stochasticity of their policy and the environment to their advantage. In the first contribution, we delved into the unsupervised reinforcement learning setting, demonstrating how an agent can learn which aspects of the environment it can control independently, while simultaneously learning a disentangled latent representation of these factors. The second contribution targeted planning in continuous control tasks. By reframing reinforcement learning as an inference problem, we borrowed tools from the Sequential Monte Carlo literature to develop a theoretically grounded and efficient algorithm for probabilistic planning using a learned model of the world, showing how the agent can leverage SMC methods to imagine diverse solutions.

The latter two contributions analyzed the impact of gradient noise due to sampling in optimization algorithms. In our third contribution, we investigated the role of gradient noise in maximum likelihood estimation with stochastic gradient descent, exploring how the structure of the gradient noise and local curvature affect the generalization and convergence speed of the model. Lastly, our fourth contribution revisited reinforcement learning to understand the impact of sampling noise on the policy gradient algorithm. We found that sampling noise can significantly influence the optimization dynamics and the policies discovered in on-policy reinforcement learning.

The work presented in this thesis provides valuable insights into the role of stochasticity in reinforcement learning and optimization algorithms, offering a solid foundation for future research in this area.

By further examining the interplay between noise and learning, we can continue to develop more robust, adaptive, and efficient algorithms that can better handle the challenges of real-world environments. Ultimately, understanding and harnessing the power of stochasticity will bring us closer to achieving the long-term goal of creating intelligent machines capable of interacting with the world and learning from experience, just as humans and animals do.




 
 






