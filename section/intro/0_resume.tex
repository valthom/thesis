\chapter*{R\'{e}sum\'{e}}
La plupart des algorithmes modernes d'apprentissage automatique intègrent un
certain degré d'aléatoire dans leurs processus, que nous appellerons le
\emph{bruit}, qui peut finalement avoir un impact sur les prédictions du modèle. Dans cette thèse, nous examinons de plus près l'apprentissage et la planification en présence de bruit pour les algorithmes d'apprentissage par renforcement et d'optimisation.

Les deux premiers articles présentés dans ce document se concentrent sur l'apprentissage par renforcement dans un environnement inconnu, et plus précisément sur la façon dont nous pouvons concevoir des algorithmes qui utilisent la stochasticité de leur politique et de l'environnement à leur avantage.
Notre première contribution présentée dans ce document se concentre sur le cadre
de l'apprentissage par renforcement non supervisé. Nous montrons comment un
agent laissé seul dans un monde inconnu sans but précis peut apprendre quels
aspects de l'environnement il peut contrôler indépendamment les uns des autres,
ainsi qu'apprendre conjointement une représentation latente démêlée de ces
aspects que nous appellerons \emph{facteurs de variation}.
La deuxième contribution se concentre sur la planification dans les tâches de
contrôle continu. En présentant l'apprentissage par renforcement comme un
problème d'inférence, nous empruntons des outils provenant de la littérature sur
les m\'{e}thodes de Monte Carlo séquentiel pour concevoir un algorithme efficace
et théoriquement motiv\'{e} pour la planification probabiliste en utilisant un
modèle appris du monde. Nous montrons comment l'agent peut tirer parti de note
objectif probabiliste pour imaginer divers ensembles de solutions.

Les deux contributions suivantes analysent l'impact du bruit de gradient dû à l'échantillonnage dans les algorithmes d'optimisation. 
La troisième contribution examine le rôle du bruit de l'esimateur du gradient dans l'estimation par maximum de vraisemblance avec descente de gradient stochastique, en explorant la relation entre la structure du bruit du gradient et la courbure locale sur la généralisation et la vitesse de convergence du modèle. 
Notre quatrième contribution revient sur le sujet de l'apprentissage par
renforcement pour analyser l'impact du bruit d'échantillonnage sur l'algorithme
d'optimisation de la politique par ascension du gradient. Nous constatons que le
bruit d'échantillonnage peut avoir un impact significatif sur la dynamique
d'optimisation et les politiques découvertes en apprentissage par
renforcement.


  {\bfseries Mots cl\'{e}s\hspace{-3pt}: Apprentissage de repr\'{e}sentations, Contr\^{o}le par Inf\'{e}rence Probabiliste, Apprentisage Profond par Renforcement, Planification.}
                                                                                                                                                            
