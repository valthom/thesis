\section{Fundamentals of machine learning and optimization}
\label{sec:ml}

\subsection{Setting and maximum likelihood estimation}

At a high level, \emph{machine learning} is the science concerned with \emph{learning from data}. This discipline is at the crossroads between different fields such as statistics, computer science and optimization. \emph{Data} is represented in the form of a dataset $\gD_\text{train} = \{x_i\}_{i=1\dots N}$ where $x_i$ are the $N$ \emph{training examples} assumed to be sampled independently and identically distributed (i.i.d) from a distribution $p$, the true data distribution. In this context \emph{learning} means finding a \emph{function} or \emph{model} that ``fits'' the data according to some loss function $\gL$.

Mathematically speaking, this is predominantly framed as a parametric optimization problem

$$\min_\theta \frac{1}{N} \sum_{i=1}^N \gL_\theta(x_i)$$

where $\gL$ is our loss function and $\theta \in \mathbb{R}^d$ is the parameter vector we optimize over. With the advent of large neural networks, $\theta$ can be a very high dimensional vector, up to hundred of billions/ a few trillions of parameters as of late 2022~\citep{chowdhery2022palm, switch2022fedus}, in opposition with classical methods for which $\theta$ was often low dimensional (compared to the number of training examples).

Commonly, our \emph{model} is a parametric function $q_\theta$ (for instance of neural network) whose goal is to approximate $p$. A natural objective to ensure $q_\theta$ becomes closer to $p$ is to maximize the likelihood of the training examples $x_i$ sampled from $p$ under $q_\theta$, i.e maximizing $q_\theta(\gD_\text{train}) = \Pi_{i=1}^N q_\theta(x_i)$ as per the i.i.d assumption. As the logarithm is a non-decreasing function and $N$ is constant, we can choose to maximize instead
$$\max_\theta \frac{1}{N} \sum_{i=1}^N \log q_\theta(x_i)$$
It appears clearly here that maximizing the likelihood of the data is equivalent to minimizing the negative log-likelihood $\gL_\theta(\cdot) = - \log q_\theta(\cdot)$ averaged over $\gD_\text{train}$.

Note that this \textbf{maximum likelihood} (or minimum negative log-likelihood) objective can be realted  to a Kullback-Leibler divergence. By calling $\hat{p}(x) = \frac{1}{N} \sum_{i=1}^N \delta_{x_i}(x)$ the empirical distribution over the training samples, we have that

\begin{align*}
\KL(\hat{p}||q_\theta) &= \frac{1}{N} \sum_{i=1}^N \log \hat{p}(x_i) - \log q_\theta(x_i) \\
&= - H(\hat{p}) +  \frac{1}{N} \sum_{i=1}^N \gL_\theta(x_i)\\
\end{align*}
As the entropy of $\hat{p}$ does not depend on $\theta$, it appears clearly that our objective is equivalent to minimizing a divergence between the empirical distribution of $p$ and $q_\theta$.

\subsection{Generalization}
Even when maximizing likelihood, one may learn a model $q_\theta$ which is accurate on the training data but inaccurate on unseen data. %
The discrepancy between the loss on the true data distribution and the training loss is called the generalization gap
$$\gG = \E_{x\sim p}[\gL_\theta(x)] -\E_{x\sim \hat{p}}[\gL_\theta(x)]$$


The generalization gap and how to estimate it is at the core of Chapter~\ref{chapter:hfc}: in our article we show how one can build estimators of $\gG$ from the local curvature and variance of our model. 

\subsection{Stochastic gradient descent}
Now we turn back our attention to the original optimization problem $\min_\theta \frac{1}{N} \sum_{i=1}^N \gL_\theta(x_i)$. When using neural networks, we can estimate efficiently estimate $\nabla_\theta \gL_\theta$ the gradient of $\gL_\theta$ using the backpropagation algorithm~\citep{rumelhart1986learning}. Thus, we can use the \emph{gradient descent} algorithm to minimize our loss

\begin{equation}
\label{eq:gradient_descent}
    \theta_{t+1} = \theta_t - \eta \frac{1}{N} \sum_{i=1}^N \nabla_\theta \gL_{\theta_t}(x_i)
\end{equation}
where $\eta$ is a positive scalar called the learning rate. Under some smoothness assumptions on $\gL$ and for a small enough $\eta$, we can show that this algorithm will converge to a local minimum of our objective function.

This algorithm is however expensive when $N$ is very large as is common in modern machine learning. An alternative to regular gradient descent is to use \textbf{stochastic gradient descent}~\citep{robbins1951stochastic}  i.e we only use a sample, or more generally, a mini-batch of $B$ samples to compute a noisy estimate of the gradient. This trade-off between noise and complexity of computing a gradient estimate is well-understood and favorable in the regime where $N$ is large~\citep{tradeoffs2007bottou}.

\begin{equation}
\label{eq:stochastic_gradient_descent}
    \theta_{t+1} = \theta_t - \eta \frac{1}{B} \sum_{i=1}^B \nabla_\theta \gL_{\theta_t}(x_i), \quad x_1, \dots x_B \stackrel{\footnotesize{\text{i.i.d}}}{\sim} \gD_\text{train}
\end{equation}

This algorithm is the central idea behind all popular optimization algorithms used for training deep neural networks, such as Adam~\citep{kingma2014adam}.